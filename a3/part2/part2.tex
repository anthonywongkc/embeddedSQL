\documentclass{article}
\usepackage{fullpage}
\usepackage{enumitem}
\usepackage{array}
\usepackage{multirow}
\title{A3 part 2}

\begin{document}

%----------------------------------------------------------------------------------------------------------------------
\renewcommand{\labelenumiii}{\arabic{enumiii}}
\begin{enumerate}

\item[2] % ---------- q2
\begin{enumerate}
\item % ---------- q2a
Compute a minimal basis for T. In your final answer, put the FDs into alphabetical order.

\begin{itemize}
\item % ----------
We'll simplify to singleton right-hand sides and call this set S1:
\begin{enumerate} 
\item AB $\rightarrow$ C
\item AB $\rightarrow$ D
\item ACDE $\rightarrow$ B
\item ACDE $\rightarrow$ F
\item B $\rightarrow$ A
\item B $\rightarrow$ C
\item B $\rightarrow$ D
\item CD $\rightarrow$ A
\item CD $\rightarrow$ F
\item CDE $\rightarrow$ F
\item CDE $\rightarrow$ G
\item EB $\rightarrow$ D
\end{enumerate}

\item % ----------
We'll look for redundant FDs to eliminate.
\begin{table}[h!]
\centering
\begin{tabular}{ p{0.5cm} | p{4cm} | p{8cm} | p{2cm} } 
 FD & Exclude these from S1 when computing closure & Closure & Decision \\ [0.5ex] 
 \hline
 1 & 1 & AB+ = ABDCF & discard \\ 
 \hline
 2 & 1, 2 & AB+ = ABCDF & discard \\
 \hline
 3 & 1, 2, 3 & There's no way to get B without this FD & keep \\
 \hline
 4 & 1, 2, 4 & ACDE+ = ACDEFG & discard \\
 \hline
 5 & 1, 2, 4, 5 & B+ = BCDAF & discard \\
  \hline
 6 & 1, 2, 4, 5, 6 & There's no way to get C without this FD & keep \\
  \hline
 7 & 1, 2, 4, 5, 7 & B+ = BC & keep \\
  \hline
 8 & 1, 2, 4, 5, 8 & There's no way to get A without this FD & keep \\
  \hline
 9 & 1, 2, 4, 5, 9 & CD+ = CDA & keep \\
  \hline
 10 & 1, 2, 4, 5, 10 & CDE+ = CDEGAFB & discard \\
  \hline
 11 & 1, 2, 4, 5, 10, 11 & There's no way to get G without this FD & keep \\
  \hline
 12 & 1, 2, 4, 5, 10, 12 & EB+ = EBCDAFG & discard \\
\end{tabular}
\end{table}

\item % ----------
Let's call the remaining FDs S2:
\begin{enumerate} 
\item[3] ACDE $\rightarrow$ B
\item[6] B $\rightarrow$ C
\item[7] B $\rightarrow$ D
\item[8] CD $\rightarrow$ A
\item[9] CD $\rightarrow$ F
\item[11] CDE $\rightarrow$ G
\end{enumerate}

\item % ----------
Let's try reducing the LHS
\begin{enumerate} 
\item[3] ACDE $\rightarrow$ B \\
A+ = A so we can't reduce the LHS to A. \\
C+ = C so we can't reduce the LHS to C. \\
D+ = D so we can't reduce the LHS to D. \\
E+ = E so we can't reduce the LHS to E. \\
AC+ = AC so we can't reduce the LHS to AC. \\
AD+ = AD so we can't reduce the LHS to AD. \\
AE+ = AE so we can't reduce the LHS to AE. \\
CD+ = CDA . . . so we can reduce the LHS to CD. 
\item[8] CD $\rightarrow$ A \\
C+ = C so we can't reduce the LHS to C. \\
D+ = D so we can't reduce the LHS to D. \\
So this FD remains as it is
\item[9] CD $\rightarrow$ F \\
C+ = C so we can't reduce the LHS to C. \\
D+ = D so we can't reduce the LHS to D. \\
So this FD remains as it is
\item[11] CDE $\rightarrow$ G
C+ = C so we can't reduce the LHS to C. \\
D+ = D so we can't reduce the LHS to D. \\
E+ = E so we can't reduce the LHS to E. \\
So this FD remains as it is
\end{enumerate}

\item % ----------
Let's call the set of FDs that we have after reducing left-hand sides S3:
\begin{enumerate} 
\item[3'] CDE $\rightarrow$ B
\item[6] B $\rightarrow$ C
\item[7] B $\rightarrow$ D
\item[8] CD $\rightarrow$ A
\item[9] CD $\rightarrow$ F
\item[11] CDE $\rightarrow$ G
\end{enumerate}

\item % ----------
Look again in case any of the changes we made allow further simplification.
\begin{table}[h!]
\centering
\begin{tabular}{ p{0.5cm} | p{4cm} | p{8cm} | p{2cm} } 
 FD & Exclude these from S1 when computing closure & Closure & Decision \\ [0.5ex] 
 \hline
 3' & 3' & There's no way to get B without this FD & keep \\
  \hline
 6 & 6 & There's no way to get C without this FD & keep \\
  \hline
 7 & 7 & There's no way to get D without this FD & keep \\
  \hline
 8 & 8 & There's no way to get A without this FD & keep \\
  \hline
 9 & 9 & There's no way to get F without this FD& keep \\
  \hline
 11 & 11 & There's no way to get G without this FD & keep \\
\end{tabular}
\end{table}

\item % ----------
No further simplifications are possible.

\item % ----------
So the following set S4 is a minimal basis:
\begin{enumerate} 
\item B $\rightarrow$ C
\item B $\rightarrow$ D
\item CD $\rightarrow$ A
\item CD $\rightarrow$ F
\item CDE $\rightarrow$ B
\item CDE $\rightarrow$ G
\end{enumerate}
\end{itemize}

\newpage
\item % ---------- q2b
Using your minimal basis from the last subquestion, compute all keys for P.

\begin{itemize}
\item % ----------
To summarize, each attrtibutes in P:
\begin{table}[h!]
\centering
\begin{tabular}{ c | c | c | c } 
\hline
\multirow{2}{*}{Attribute} & \multicolumn{2}{c|}{Appears on} & \multirow{2}{*}{Conclusion} \\
\cline{2-3}
& LHS & RHS \\ [0.5ex] 
\hline
H & - & - & must be in every key \\
\hline
E & $\surd$ & - & must be in every key \\
\hline
A, F, G & - & $\surd$ & is not in any key \\
\hline
B, C, D & $\surd$ & $\surd$ & must chek \\
\hline
\end{tabular}
\end{table}

\item % ----------
Therefore, we only need to consider the combination of B, C, D\\
BEH+ = BEHCDAFG. So BEH is a key.
CEH+ = CEH. This is not a key
DEH+ = DEH. This is not a key

\item % ----------
The keys for P are BEH
\end{itemize}

\item % ---------- q2c
Employ the 3NF synthesis algorithm to obtain a lossless and dependency-preserving decomposition of
relation P into a collection of relations that are in 3NF.

 \begin{itemize}
\item % ----------
Let's call the revised FDs S5:
\begin{enumerate} 
\item B $\rightarrow$ CD
\item CD $\rightarrow$ AF
\item CDE $\rightarrow$ BG
\end{enumerate}

\item % ----------
The set of relations that would result would have these attributes: \\
R1(BCD), R2(ACDF), R3(BCDEG)

\item % ----------
Since the attributes BCD occur within R3, we don't need to keep the relation R1.

\item % ----------
BEH is a key of P which is not in the set of relations so we need to add another relation that includes a key. 

\item % ----------
So the final set of relations is: \\
R2(ACDF), R3(BCDEG), R4(BEH)
\end{itemize}

\item % ----------
Does your schema allow redundancy?

\begin{itemize}
\item % ----------
We need to find out the projection of the FDs onto each relation to ensure the FDs do not violate BCNF and therefore redundancy is allowed.

\item % ----------
Clearly, CD $\rightarrow$ AF will project onto the relation R4 and it is not a superkey of this relation.

\item % ----------
So yes, these schema allows redundancy.
\end{itemize}


\end{enumerate}

\end{enumerate}

\end{document}



